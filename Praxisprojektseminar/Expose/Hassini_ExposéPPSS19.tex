\documentclass[a4paper, 12pt]{scrartcl}

\usepackage[ngerman]{babel}  % Deutsche Einstellungen
\usepackage[utf8]{inputenc} %uft-8 Eingabe
\usepackage[T1]{fontenc}
\usepackage{csquotes}
\usepackage{graphicx}

\begin{document}


\begin{titlepage}
	
	\begin{center}
		
		% Logo der Technische Hochschule Köln
		% Kann auch in dieser Form in Schwarz/Weiß ausgedruckt werden; Graustufen sollten der .tif Version entsprechen
		\begin{figure}[!ht]
			%	\centering
			\includegraphics[width=0.26\textwidth]{images/THlogoheader.pdf}
		\end{figure}
		
		\vspace{0.8cm}
		
		%Deutscher Titel
		\begin{rmfamily}
			\begin{huge}
				\textbf{Therapieoptimierung\\ für Diabetiker}\\	
			\end{huge}
			\vspace{0.5cm}
			\begin{LARGE}
				Recherche, Analyse und\\ Lösungsentwicklung\\
			\end{LARGE}
		\end{rmfamily}
		
		\vspace{1.6cm}
		
		%Englischer Titel
		% \begin{rmfamily}
		% \textbf{\LARGE Title in English}\\
		% \large with a very\\long subtitle\\
		% \normalsize
		% \end{rmfamily}
		
		% \vspace{1.2cm}
		
		%Bachelorarbeit 
		\begin{LARGE}
			\begin{scshape}
				Exposé zum Praxisprojekt\\[0.8em]
			\end{scshape}
		\end{LARGE}
		
		%ausgearbeitet von...
		\begin{large}
			ausgearbeitet von\\ 
			\vspace{0.2cm}
			\begin{LARGE}
				Sami Hassini\\
			\end{LARGE}
		\end{large}
		
		\vspace{1.0cm}
		
		
		%vorgelegt an der...
		\begin{large}
			vorgelegt an der\\ 
			\vspace{0.2cm}
			\begin{scshape}
				Technischen Hochschule Köln\\
				Campus Gummersbach\\
				Fakultät für Informatik und\\
				Ingenieurwissenschaften\\
			\end{scshape}
		\end{large}
		
		\vspace{0.4cm}
		
		%im Studiengang...
		\begin{large}
			im Studiengang\\ 
			\vspace{0.2cm}
			\textsc{Medieninformatik}
		\end{large}
		
		
		\vspace{1.0cm}
		
	
		
		%Ort, Monat der Abgabe
		\begin{large}
			Köln, 10. April 2019
		\end{large}
		
	\end{center}

\end{titlepage}
	
	\newpage
	\thispagestyle{empty}
	
	

	
			\section{Problemfeld und Kontext}
			Diabetes mellitus ist eine Stoffwechselerkrankung in der Bauchspeicheldrüse, bei der die Aufnahme von Glukose aus dem Blut in die Körperzellen unterbunden wird, wodurch erhöhte Blutzuckerwerte entstehen. \\
			Ein guter Blutzuckerwert liegt im Bereich von 80 bis 120 Milligramm pro Deziliter. Der Körper speichert Zucker im Blut, Leber und Körperzellen. Nach der Essensaufnahme werden Kohlenhydrate in Glucose umgewandelt und dieses gelangt folglich in Blut und Leber. Die Leber bietet eine Zuckerspeicherung, die als Reserve dient und aufgebraucht wird, wenn die körperliche Bewegung und der Energieverbrauch des Körpers hoch ist. Insulin wird von Inselzellen in der Bauchspeicheldrüse produziert und sorgt für den Transport des Zuckers aus dem Blut und Leber in die Körper- und von dort in die Muskelzellen. Insulin dient metaphorisch als Schlüssel für die Muskelzellen, die das Schloss darstellen, sodass man von einem Schlüssel-Schloss-Prinzip reden kann. Neben den hohen Blutzuckerwerten kann ein Diabetiker auch zu niedrige Blutzuckerwerte haben. Dies wird durch Sport oder zu viel Insulin verursacht.\\
			Eine Überzuckerung nennt man Hyperglykämie und bedeutet „zu viel Zucker im Blut“. Dies kann zur einer Ketoacidose, Übersäuerung des Blutes, führen. Hyperglykämien sind immer ernst zunehmen und müssen konsequent behandelt werden. Kommt es tatsächlich zu einer Ketoacidose, in der Ketone in die Blutbahn und in den Urin gelangen, könnte man bei Nichtbehandlung ins Koma fallen oder sogar sterben. 
			Die Ketoacidose tritt meist bei Werten ab 200mg/dl über mehrere Stunden auf und ist die gefährlichste Akutkomplikation des Diabetes. Der Großteil der Todesfälle durch Diabetes ereignen sich durch Ketoacidosen und folglich Hirnödem. 
			Das gefährlich bei einer Ketoacidose sind die Ketone in Blut und Urin. Bei Glucosemangel in Muskel- und Körperzellen wird Glukagon als Hunger-Signal der Zelle ausgeschüttet. Dieses Glukagon sorgt dafür, dass die Zuckerreserven aus der Leber in die Blutbahn gelangen und somit der Blutzucker steigt. Auch dieser Zucker gelangt nicht in die Körperzellen, sodass der Körper weiter Glucose in die Blutbahn befördern möchte. Die Fettreserven werden verbrannt, wodurch freie Fettsäuren entstehen und Ketonkörper als Abfallprodukt produziert werden. Ketone sorgen für eine Übersäuerung des Blutes und scheiden über die Atmung und den Urin aus. Zudem kommt es zu einer Austrocknung des Körpers, da dieser sich von Ketone durch Wasserlassen reinigen möchte. Folglich kann es durch austrocknen der Hirnzellen zur Bewusstseinsschwäche und somit zum Koma kommen. In dieser Phase schwebt man in Lebensgefahr. Eine Überzuckerung wird durch die Einnahme von Insulin vermieden.\\
			Bei einer Hypoglykämie hat man zu wenig Zucker im Blut. Dies tritt auf, wenn dem Körper zu viel Insulin zugeführt oder keine Kohlenhydrate über einen längeren Zeitraum aufgenommen wurden. Von einer Hypoglykämie oder Unterzuckerung spricht man, wenn der Blutzucker unter 80mg/dl liegt. Sinkt der Blutzuckerwert weiter gegen 0mg/dl, steigt die Gefahr der Bewusstlosigkeit. Diese sorgt für Muskelzuckungen und hält solange an, bis der Körper Adrenalin ausstößt. Adrenalin hat eine blutzuckererhöhende Wirkung. Um aus der Unterzuckerung zu gelangen, ist es notwendig schnelle Kohlenhydrate wie Traubenzucker oder Orangensaft zu sich zu nehmen.\\
			
			
			Laut der Weltgesundheitsorganisation (WHO) hat sich die Zahl der Diabetiker seit 1980 weltweit auf etwa 422 Millionen nahezu vervierfacht. Diabetes mellitus ist eine Krankheit, die mittlerweile überall auf der Welt und bei jeder Altersgruppe auftritt. 
			Um den Diabetes mellitus in den Griff zu bekommen, ist es notwendig als Erkrankter 4-6 mal am Tag den Blutzucker zu messen und bei jeder Einnahme von Kohlenhydraten Insulin zu spritzen. Gerade im Kindes- und Jugendalter lässt sich dies nicht leicht umsetzen. Erkrankte Kinder und Jugendliche können gerade in der Phase der Pubertät die nötige Eigeninitiative zum Blutzucker Messen nicht aufbringen. Darunter leiden sehr oft die Blutzuckerwerte und folglich werden Organe wie Niere, Leber oder die Augen beschädigt.
			Auch Erwachsene haben oft einen strammen Zeitplan und im Alltag nicht immer die notwendige Zeit, dass Messgerät in die Hand zu nehmen, sich zu pieksen und zu warten bis das Gerät den Blutzuckerwert ausgibt. Zudem müssen Werte zur Analyse dokumentiert werden und in sogenannte Tagebücher eingetragen werden. Zumal bei der Blutzuckermessung und Insulininjektion immer eine bestimmte Hygiene beachtet werden muss. Somit ist das Messen und Spritzen an einem Tag, an dem ein Diabetiker einen längeren Zeitraum unterwegs ist, fast unmöglich. Und auch in der Nacht, während der Schlafphase, entstehen Zeiträume von mehreren Stunden, in denen keine Blutzuckerwerte erfasst werden können und so nicht behandelt werden. Dies beeinträchtigt jeden Diabetiker in jeder Altersgruppe.

			\section{Ziele}
				Ziel dieses Projektes ist es einen geeigneten Anwendungsbereich eines möglichen Systems zur Optimierung der Lebensqualität eines Diabetikers, anhand von Literatur- und Marktrecherchen, sowie der Analyse von Umfragen und Interviews mit Stakeholdern, zu konzipieren. Dieser Anwendungsbereich sollte erkennbare Alleinstellungsmerkmale enthalten und den Anforderungen der Stakeholder, welche aus der Evaluation erforscht werden, entsprechen. Darüber hinaus soll das zu konzipierende System zur Optimierung der Blutzuckerwerte eines Diabetikers beitragen und eine transparente, einfache und zeitsparende Dokumentation der Blutzuckerwerten ermöglichen. Das Projekt soll auf folgende Forschungsfragen und Unterfragen Antworten finden:
				\begin{itemize}
					 
					\item Welche technischen Hilfsmittel steigern die Lebensqualität eines Diabetikers?
					\item Was ist Diabetes Mellitus?
					\item Wie lebt ein Diabetiker aktuell?
					\item Wie sieht der Stand der aktuellen technischen Hilfsmittel des Diabetes aus?
					\item Wie zufrieden sind die Diabetiker mit heutigen technischen Hilfsmitteln?
					\item Was fehlt ihnen? Und welche Vorstellungen haben sie von den zukünftigen Hilfsmitteln?
					\item Wie werden zukünftige technische Hilfsmittel der Diabetes aussehen?
				\end{itemize}
			
			\section{Aufgabenstellung}
				Die Recherche und das Konzept zu einer Lösung zur Optimierung der Lebensqualität eines Diabetikers unter Einbindung von Evaluation durch Diabetikern und weiteren Stakeholdern. 
			
			\section{Lösungsansätze}
				Erste Ansätze zu Problemlösung wären zum einen eine umfangreiche Domänenrecherche und die Analyse möglicher Stakeholder. Des Weiteren ist es notwendig, eine Marktrecherche durchzuführen, um die aktuellen Systeme auf dem Markt zu analysieren und mögliche Alleinstellungsmerkmale zu erkennen. Darüber hinaus sollen Umfragen und Interviews mit Diabetikern durchgeführt werden um deskriptive und mögliche präskriptive Modelle zu konzipieren. Durchgeführte Umfragen und Interviews müssen abschließend ausgewertet und analysiert werden. Anhand der zuvor durchgeführten Recherchen und Analysen wird der Anwendungsbereich, welcher in der Bachelorarbeit implementiert und dann evaluiert werden soll, definiert und die Abgrenzung und der Bezug des Paxisprojektes zur Bachelorarbeit bestimmt.
			
			\section{Chancen und Risiken}
				Im Idealfall können alle gesetzten Ziele erreicht werden und eine Grundlage für eine mögliche Bachelorarbeit und damit eine Implementierung des Projektes konzipiert werden.				
				Dabei sollte die Komplexität der Domäne und des Projektes bedacht werden, um folgenschwere Schäden am späteren Ergebnisse zu vermeiden. Es muss bereits frühzeitig darüber nachgedacht werden, welche Priorität einzelne Aufgabenbereiche für das Projekt haben und folglich wie viel Zeitaufwand für diese eingeplant werden muss. Hier sollte ein Projektplan mit einzelnen Projektphasen und –aufgaben erstellt werden, um ein Zeitmanagement zu garantieren und zeitliche Probleme zu vermeiden.
				Zudem müssen im Vorfeld bereits mögliche Personen und Firmen für Interviews und Umfragen angefragt werden, um beim Zeitpunkt der Durchführung der Interviews keine Zeit verloren wird. Um ausreichend Material zur Analyse der Umfragen zu erhalten, sollte in Krankenhäusern, bei Ärzten, Diabetikern und Diabetologen angefragt werden, ob Interviews und Umfragen möglich wären.
			
			\section{Ressourcen}
				Als Technologie oder Kooperationspartner käme Dexcom G6 in Frage. Dexcom ist eine Firma, welche kontinuierliche Blutzuckermessungen vornimmt und anhand eines Sensors bis zu 266 Blutzuckerwerte am Tag ermöglicht. Zudem bieten sie eine Schnittstelle zu den Datenbanken der Blutzuckerwerte der Benutzer. Mit diesen Blutzuckerwerten könnte im Praxisprojekt gearbeitet werden.

			
			\section{Motivation}
			Ich habe dieses Thema für mein Praxisprojekt gewählt, da ich selber seit über 10 Jahren Typ-1 Diabetiker bin und ich die Defizite aktueller Hilfsmittel aus Erfahrung kenne. Als Medieninformatiker gehört das Entwickeln von interaktiven Systemen zu Fachgebiet. Mit diesen Kenntnissen und mit meinen eigenen Erfahrungen, sowie den Recherchen und Analysen des Projektes, möchte ich eine benutzerfreundliche und effiziente Begleit-Anwendung für Diabetiker konzipieren und entwickeln. 

		
	
	

\end{document}



